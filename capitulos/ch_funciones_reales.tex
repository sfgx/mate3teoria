\chapter{Funciones Reales de Varias Variables}

\begin{informacion}[colbacktitle=myblue]{\Large Contenidos}
\begin{cirlist}
    \item Definición del conjunto $\Rn$.Representación gráfica de 
    subconjuntos de $\R^{2}$. Dominio y rango. Gráfica de funciones de $2$ 
    variables independientes y de sus dominios. Definición de curvas de nivel. 
    Propiedades. Límite y continuidad de funciones de dos variables.
    \item 	Entornos de un punto. Puntos frontera y Puntos interiores de un 
    conjunto. Conjuntos: acotados, abiertos, cerrados, convexos.
    \item	Definición de Funciones reales de varias variables independientes. 
    Dominio y rango. Gráfica de funciones de 2 variables independientes y de 
    sus dominios.
    \item Definición de curvas de nivel. Propiedades.
    \item Límite de Funciones de dos Variables: Definición intuitiva y 
    rigurosa de límite o límite doble. Límites sucesivos o iterados. Límite 
    Radial.
    \item Continuidad de una función de dos variables. Continuidad en un 
    punto. Propiedades.
\end{cirlist}
\end{informacion}


\begin{informacion}[colbacktitle=mostaza]{\Large Objetivos}
Al finalizar el estudio de esta unidad temática serás capaz de:
\begin{cirlist}
    \item Identificar y analizar funciones de varias variables.
    \item Graficar curvas de nivel como método para representar geométricamente 
    funciones de dos variables
    \item Interpretar geométricamente los conceptos de límite y calcular 
    limites.
    \item Determinar la continuidad y clasificar las discontinuidades de una 
    función.
    \item Analizar la continuidad de funciones en situaciones problemáticas.
\end{cirlist}
\end{informacion}


\begin{informacion}[colbacktitle=morado]{\Large Conocimientos Previos}
Para los objetivos propuestos necesitarás revisar los siguientes contenidos: 
\begin{cirlist}
    \item Funciones reales de una variable real. Definición. Dominio y rango.
    \item Funciones algebraicas simples: función lineal, cuadrática y racional.
    \item Funciones exponencial, logarítmica y trigonométricas. 
    \item Límite y continuidad de funciones de una variable real
\end{cirlist}
\end{informacion}


\section{Conjuntos de $\Rn$. Representación gráfica de subconjuntos de $\Rn$ }

Para desarrollar los conceptos de esta unidad definiremos algunos conceptos 
importantes y necesarios para el desarrollo de los contenidos de esta unidad, 
entre ellos: distancia entre dos puntos, vecindario, punto interior y punto 
frontera ya utilizados en matemática pero que ahora los definiremos para puntos 
de $\Rn$.

\begin{definicion}[Distancia entre dos puntos]
Sean los puntos $P(x_1, x_2, \dots, x_n)$  y  $Q(y_1,y_2,\dots, y_n)$. La 
distancia entre $P$ y $Q$ está dada por:

$$d = \overline{PQ} = \sqrt{(x_{1}-y_{1})^{2}+(x_{2}-y_{2})^{2}
                        + \cdots +(x_{n}-y_{n})^{2}}$$
\end{definicion}

\begin{multicols}{2}
{\bf Observación:} si $P \, y \, Q \, \in \R^{2}$\ la definición de 
distancia entre dos puntos quedaría: Si $P$ es un punto fijo de coordenadas 
$(a, b)$ y $Q$ un punto variable de coordenadas $(x, y)$, 
$d=\sqrt{(x-a)^{2}+(y-b)^{2}}$, entonces $Q$ es un punto que pertenece a la 
circunferencia con centro en $P(a, b)$ y radio d como se muestrea en la 
figura \ref{fg_distancia}.

\begin{center}
    \includegraphics[scale= 2]{imagenes/distancia_puntos.png}
        \captionof{figure}{Distancia entre dos puntos de $\R^{2}$}
        \label{fg_distancia}
\end{center}
\end{multicols}

\subsection*{Entornos de un punto de $\Rn$}

\begin{definicion}[Vecindario abierto de centro $P_1$ y radio $r$]
Sea $P_1 = (y_1, y_2,\dots, y_n)$ entonces el vecindario de centro $P_1$ y radio
$r$ es el conjunto de todos los puntos P de $\Rn$ tales que su distancia al 
punto $P_1$ sea menor que $r$.

Simbólicamente:

\begin{equation*}
    \begin{split}
    V(P_1,r) & = \left\{ P \in \Rn \sth d(P,P_1) < r \right\} \\
             & = \left\{ P \in \Rn \sth \sqrt{(x_1 - y_1)^2 + (x_2 - y_2)^2 +
             \cdots + (x_n - y_n)^2 < r}
             \right\}
    \end{split}
\end{equation*}
\end{definicion}


\textbf{Observaciones}

\begin{itemize}
    \item En el caso de $n=2$ tenemos que el vecindario de centro $P(a,b)$ y 
    radio $h$ es

    $$V(P,h) = V((a,b),h)  =\left\{(x,y)/d=\sqrt{(x-a)^{2}+(y-b)^{2}}<h\right\}$$

    O sea: 
 
    $$(x,y) \sth \sqrt{(x-a)^{2}+(y-b)^{2}}<h \Rightarrow (x-a)^{2}+(y-b)^{2}<h^{2}$$


    \begin{multicols}{2}
    Geométricamente el vecindario $V(P, h)$ es el conjunto de los puntos del 
    plano interiores al círculo con centro en $P(a, b)$ y radio $h$ que se 
    muestra en la figura \ref{fg_vec_ab_2d}

    \begin{center}
        \includegraphics[scale=1.5]{imagenes/vecindario_abierto_2d.png}
        \captionof{figure}{Vecindario en $\Rn$}
        \label{fg_vec_ab_2d}
    \end{center}
    \end{multicols}

    \item En el caso de $n=3$ tenemos que, el vecindario de centro $P(a,b,c)$ y 
    radio $h$ es:

      $$V(P,h)=V((a,b,c),h)=\left\{(x,y,z)/d=\sqrt{(x-a)^{2}+(y-b)^{2}+(z-c)^{2}}<h\right\}$$
    o sea,

    $$ (x,y,z)\sth \sqrt{(x-a)^{2}+(y-b)^{2}+(z-c)^{2}}<h\Rightarrow(x-a)^{2}+(y-b)^{2}+(z-c)^{2}<h^{2}$$

    \begin{multicols}{2}
    Geométricamente el vecindario $V(P, h)$ es el conjunto de los puntos del 
    espacio  interiores a una esfera con centro en $P(a, b,c)$ y radio $h$ como  
    se muestra en la figura \ref{fg_vec_ab_3d}

    \begin{center}
    \includegraphics[scale=1.5]{imagenes/vecindario_abierto_3d.png}
    \captionof{figure}{Vecindario en $\R^{3}$}
    \label{fg_vec_ab_3d}
    \end{center}
    \end{multicols}
\end{itemize}

\begin{definicion}[Vecindario reducido del punto $P_1$ y radio $h$]
 Sea $P_1 = (y_1,y_2, \dots, y_n)$ entonces el vecindario reducido de centro $P_1$ 
 y radio $r$ es el vecindario que resulta de excluir su centro.

Simbólicamente:
\begin{equation*}
    \begin{split}
        V^{*}(P_{1},r) & = V(P_{1},r)-\left\{P_{1}\right\} 
                         = \left\{P \in \Rn \sth 0< d(P,P_1)<r \right\} \\
                       & = \left\{P \in \Rn \sth 0<\sqrt{(x_1-y_1)^{2}+(x_2-y_2)^{2}
                           +\cdots + (x_n-y_n)^{2}} < r\right\}
    \end{split}
\end{equation*}
\end{definicion}   

\textbf{Observación}:

En el caso de $n=2$ tenemos que el vecindario reducido de centro $P(a,b)$ y 
radio $h$ es

$$V^{*}(P,h)=V^{*}((a,b),h)=\left\{(x,y) \sth 0<\sqrt{(x-a)^{2}+(y-b)^{2}}<h\right\}$$

o sea 

$$ (x,y) \sth 0<\sqrt{(x-a)^{2}+(y-b)^{2}}<h\Rightarrow 0<(x-a)^{2}+(y-b)^{2}<h^{2}$$

\begin{multicols}{2}
Geométricamente el vecindario  reducido de centro $P(a,b)$ y radio $h$ es el 
conjunto de los puntos del plano interiores al círculo con centro en $P(a, b)$ 
y radio $h$, excluido el centro $P(a,b)$ como se muestra en la figura 
\ref{fg_vec_red_2d}.

\begin{center}
    \includegraphics[scale=2]{imagenes/vecindario_reducido_2d.png}
    \captionof{figure}{Vecindario reducido de un punto de $\R^2$}
    \label{fg_vec_red_2d}
\end{center}
\end{multicols}

\begin{definicion}[Punto interior]
Un punto $P$ perteneciente a un conjunto $D \subset \Rn$, se dice que es 
\textbf{interior} a D si existe un $V(P, h)$ totalmente incluido en $D$. 

O sea:

$$P \text{ es interior a } D \text{ si existe } V(P,h) \subset D$$
\end{definicion}

\begin{multicols}{2}
En la Figura \ref{fg_pto_int} se muestra un conjunto $D$ en el cual el punto $P$
es interior a $D$, mientras que el punto $H$ no es interior a $D$.

Los puntos interiores de una región conforman (como conjunto) el 
\textbf{interior de la región}.

\begin{center}
    \includegraphics[scale=1]{imagenes/punto_interior.png}
    \captionof{figure}{Punto interior}
    \label{fg_pto_int}
\end{center}
\end{multicols}

\begin{definicion}[Punto frontera]
    Un punto $P$ se dice punto frontera del conjunto  $D \subset \Rn$, el cual 
    puede o no pertenecer, si cualquier $V(P, h)$ contiene algún punto de $D$ 
    y algún punto de fuera de $D$.
\end{definicion}

\begin{multicols}{2}
Si $n=2$ , o sea $D \subset \R^2$ en la figura \ref{fg_pto_fron} se muestra un 
punto frontera

\begin{center}
    \includegraphics[scale=2]{imagenes/punto_frontera.png}
    \captionof{figure}{Punto frontera}\label{fg_pto_fron}
    \end{center}
\end{multicols}

\begin{multicols}{2}
Al conjunto de todos los puntos frontera del conjunto D se lo denomina 
\textbf{la frontera de $D$}.

\begin{center}
    \includegraphics[scale=1.5]{imagenes/frontera.png}
    \captionof{figure}{Gráfica de la frontera de $D$}
    \label{fg_frontera}
    \end{center}
\end{multicols}

\begin{definicion}[Conjunto abierto]
Un conjunto $ A \subset \Rn$  es un conjunto abierto si, para cada punto $P$ 
de $A$ se puede encontrar un vecindario $V(P, h)$ incluido en $A$. O sea, 
si todos los puntos de $A$ son interiores.
\end{definicion}

\begin{definicion}[Conjunto cerrado]
Un conjunto $ A \subset \Rn$  es un conjunto cerrado si todos los puntos 
frontera de $A$ pertenecen al conjunto $A$. 
\end{definicion}

\begin{definicion}[Conjunto acotado]
Un conjunto $ A \subset \Rn$  es un conjunto acotado si existe algún número 
real $h>0$ y un punto $ P \in \Rn \sth A \subset  V(P,h)$.
\end{definicion}

A continuación mostramos algunos ejemplos de los conjuntos definidos 
anteriormente.

\begin{multicols}{2}
La gráfica de la figura \ref{fg_abierto_acotado} muestra un conjunto abierto 
y acotado.

\begin{center}
    \includegraphics[scale=0.25]{imagenes/abierto_acotado.png}
    \captionof{figure}{Conjunto abierto y acotado}
    \label{fg_abierto_acotado}
    \end{center}
\end{multicols}

\begin{multicols}{2}
La gráfica de la figura \ref{fg_cerrado_acotado} muestra un conjunto cerrado 
y acotado.

\begin{center}
    \includegraphics[scale=0.25]{imagenes/cerrado_acotado.png}
    \captionof{figure}{Conjunto cerrado y acotado}
    \label{fg_cerrado_acotado}
    \end{center}
\end{multicols}

\begin{multicols}{2}
La gráfica de la figura \ref{fg_cerrado_no_acotado} muestra un conjunto cerrado y 
no acotado.

\begin{center}
    \includegraphics[scale=0.25]{imagenes/cerrado_no_acotado.png}
    \captionof{figure}{Conjunto cerrado y no acotado}
    \label{fg_cerrado_no_acotado}
    \end{center}
\end{multicols}

\begin{definicion}[Conjunto convexo]
Un conjunto $ A \subset \Rn$ es un conjunto convexo si: dados dos puntos 
cualesquiera $X$ e $Y$ de $A$ el segmento que los une está incluido en  
conjunto $A$. Simbólicamente,

$$ A \subset \Rn \text{ es un conjunto convexo si } 
\overline{XY} \subset A;\;\; \forall X \in A \text{ y } \forall Y \in A $$
 
\end{definicion}

Esta condición se la podría expresar también de la siguiente manera:

Un conjunto $ A \subset \Rn$  es un conjunto convexo si 
$\lambda X + (1-\lambda )Y \in A,\; \forall X\in A;\; \forall Y\in A;\; 
\lambda \in [0,1]$.


\begin{nota}
    \begin{minipage}{0.60\textwidth}
    $\overline{XY} =\left\{Z \sth \exists\, \lambda  \in [0,1]:Z=\lambda X+(1-\lambda ) Y\right\}$
    \end{minipage}
    \hfill
    \begin{minipage}{0.35\textwidth}
        \centering
        \includegraphics[scale=0.35]{imagenes/segmento_xy.png}
        \captionof{figure}{Gráfica del segmento $\overline{XY}$}
        \label{fg_segmento_xy}
    \end{minipage}
\end{nota}


\section{Introducción al concepto de funciones}

\subsection*{Función Real de $n$ variables independientes}

\subsection*{Función Real de $2$ variables independientes}

\subsection*{Gráfica de una función de $2$ variables}

\subsection*{Curvas de nivel de una función de dos variables}

\subsection*{Límites de funciones de dos variables}

\subsection*{Continuidad de funciones de dos variables}



