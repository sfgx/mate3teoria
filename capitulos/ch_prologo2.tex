
\begin{prologo}

\textbf{A LOS ESTUDIANTES QUE INICIAN EL CURSADO DE MATEMÁTICA III}

Los Docentes de la Cátedra Matemática III de la Facultad de Ciencias Económicas 
de la UNT, elaboraron este Cuaderno que contiene una selección de ejercicios y 
problemas para el cursado del espacio curricular en el primer cuatrimestre del 
año 2026. En cada unidad se incluyen los conceptos teóricos básicos necesarios 
para lograr la dinámica adecuada en el desarrollo de los ejercicios propuestos 
en las clases prácticas.

Recuerda que, para lograr un mejor rendimiento académico, deben realizar, en lo 
posible para cada clase, lo siguiente

\begin{enumerate}
    \item Lectura individual y/o grupal del material bibliográfico y de los 
    apuntes de clases.
    \item Análisis y discusión de las dificultades presentadas con su grupo de 
    compañeros.
    \item Análisis por parte del docente a cargo de las situaciones presentadas 
    por los alumnos
\end{enumerate}

Deben tener en cuenta que  un texto de matemáticas es muy diferente a una 
novela, un periódico o hasta otro libro. Y al momento de comenzar a leerlo, 
deberán releer un párrafo varias veces antes de entenderlo. Lo que no debe 
desalentarlos.

\textbf{Pongan especial atención a los ejemplos y resuélvanlos} con lápiz y 
papel a medida que los lean y a continuación, hagan los ejercicios 
relacionados. El plantearse resolver muchos problemas ayuda a entender el 
``mecanismo'' de cada uno de ellos. Esto les facilita ``ver'' de un grupo de 
problemas, cuáles son los similares o intuir por donde deben comenzar para 
resolverlos.

Hay muchas formas de resolver un problema, y muchas más de entenderlo. Por lo 
que les pedimos que no se desalienten cuando comienzan las dificultades.

\end{prologo}

\vspace{2cm}

\epigraph{Dios creó los números. El hombre todo lo demás.}
        {\textit{Leopold Kronecker, \\ matemático del siglo XIX.}}