%%%%%%%%%%%%%%%%%%%%%%%%%%%%%%%%%%%%%%%%%%%%%%%%%%%%%%%%%%%%%%%%%%%%%%%%%%%%%%%
%%%%% PAQUETES %%%%%
%%%%%%%%%%%%%%%%%%%%%%%%%%%%%%%%%%%%%%%%%%%%%%%%%%%%%%%%%%%%%%%%%%%%%%%%%%%%%%%
\usepackage[spanish]{babel}
\usepackage[utf8]{inputenc}
\usepackage{microtype}
\usepackage{multicol}
\usepackage{float}
\usepackage{multirow}
\usepackage{wrapfig} 
\usepackage{lscape}
%\usepackage[autostyle, spanish = mexican]{csquotes}
\usepackage{xcolor}
\usepackage{pstricks}
\usepackage[many]{tcolorbox}
\usepackage{varwidth}
\usepackage{tikz,tkz-tab}
\usepackage{tikzpagenodes}
\usepackage{xargs}  
\usepackage[hypcap=false]{caption}
%\usepackage[breaklinks,colorlinks=true, linkcolor=azulF, 
%            citecolor=azulF, urlcolor=azulF]{hyperref}
\usepackage{amsmath,amssymb,amsfonts,amsthm}%
\usepackage{latexsym,cancel,stackrel,stmaryrd}
%\usepackage[ruled,vlined,lined,linesnumbered,algochapter]{algorithm2e}
\usepackage{framed}
\usepackage{titletoc}
\usepackage{calc}
\usepackage{colortbl} 
\usepackage{tabularx}
\usepackage{array}
\usepackage{wasysym}
\usepackage{xtab}
\usepackage{booktabs}
\usepackage[shortlabels]{enumitem}
%\usepackage{textgreek}
\usepackage{esvect}
\usepackage{fancyhdr}
\usepackage[explicit]{titlesec}                                                 
\usepackage{epigraph}
\usepackage{xpatch}
\usepackage{answers}
\usepackage{xstring}
\usepackage{listings}
\usepackage{fancyvrb}
\usepackage{array}
\usepackage[absolute]{textpos}
\usepackage{pdfpages}

\usepackage[fixed]{fontawesome6}

\usepackage[a4paper, centering,
            text={17cm,26cm},
            showframe=false]{geometry}

\usepackage{rotating}

%-------------------------------------------------------------------------------
% Título
%-------------------------------------------------------------------------------
\newcommand*{\titulo}[4]{\begingroup%
\raggedleft 
\vspace*{\baselineskip} % Espacio en blanco en la parte superior de la página
{#1}\\[0.167\textheight] % Autor
{#2}\\[\baselineskip] % pre-título
{#3}\\ % Título
{#4}\par % Descripción adicional
\bigskip

\vfill % Espacio en blanco entre el bloque de título y "la editorial"

{\raggedright
%\begin{minipage}[c]{0.08\textwidth}
%\raisebox{-2.0cm}{\includegraphics[width=1.4cm]{images/fau}}
% \end{minipage}
\hfill\begin{minipage}[t]{0.9\textwidth}
{\color{gray}
 \fhv{9}{Serie Cuadernos }\\
 \fhvb{9}{\color{azulF}Notas téoricas para Matemática III-FACE-UNT.}
  \fntg[pag][8]{\color{grisF}
  }}
\end{minipage}          
}%raggedright
\vspace*{3\baselineskip} % Espacio en blanco antes del final de página
\endgroup}
% Fin Titulo--------------------------------------------------------------------


%-------------------------------------------------------------------------------
% Cabeceras
%-------------------------------------------------------------------------------
\pagestyle{fancy}
 \renewcommand{\chaptermark}[1]{\markboth{#1}{#1}}
 \fancyhf{} % borra cabecera y pie actuales
 \fancyhead[R]{\bfseries \helv\thepage} %Left Even page - Right Odd page

 \renewcommand{\headrulewidth}{0pt} % Sin raya. Con raya?: cambiar {0} por {0.5pt}
 \renewcommand{\footrulewidth}{0pt}
 \setlength\headheight{14.5pt}
 \fancyheadoffset[R]{0.0cm} %Numeración de página en el borde de la página
  \addtolength{\headheight}{0.5pt} % espacio para la raya
 \fancypagestyle{plain}{%
 \fancyhead{} % elimina cabeceras y raya en páginas "plain"
 \renewcommand{\headrulewidth}{0pt} }
% Fin cabeceras ---------------------------------------------------------------

%------------------------------------------------------------------------------
% Copypright, ISBN, ...
%------------------------------------------------------------------------------
\def\copyrightpage{\thispagestyle{empty}%
\vbox to\textheight\bgroup\vfill\obeylines\obeyspaces\xcopyrightpage}

\def\xcopyrightpage#1#2\end#3{\scriptsize\parindent=0pt
Copyright\copyright{#1} 
\vskip40pt
#2\vskip200pt\egroup\endgroup}
\let\endcopyrightpage\relax
% Fin Copyright-----------------------------------------------------------------

%----------------------------------------------------------------------------------------
% CAPITULO Estilo simple
%----------------------------------------------------------------------------------------
\newlength\ChapWd
\settowidth\ChapWd{\huge\chaptertitlename}

\titleformat{\chapter}[display]
  {\normalfont\filcenter\sffamily}
  {\tikz[remember picture,overlay]
    {
    \node[fill=myblue,font=\fontsize{60}{72}\selectfont\color{white},anchor=north east,minimum size=\ChapWd] 
      at ([xshift=-15pt,yshift=-15pt]current page.north east) 
      (numb) {\thechapter};
    \node[rotate=90,anchor=south,inner sep=0pt,font=\LARGE] at (numb.west) {\chaptertitlename};
    }
  }{0pt}{\fontsize{26}{31}\selectfont\color{myblue}#1}[\vskip-15pt\rule{15cm}{.1pt}]
\titlespacing*{\chapter}{0pt}{10pt}{15pt}

\makeatletter
\xpatchcmd{\ttl@printlist}{\endgroup}{{\noindent\color{myblue}\rule{\textwidth}{1.5pt}}\vskip30pt\endgroup}{}{}
\makeatother

\newcommand\DoPToC{%
\startcontents[chapters]
\printcontents[chapters]{}{1}{\noindent{\color{myblue}\rule{\textwidth}{1.5pt}}\par\medskip}%
}

%-------------------CONTENIDO -----------------------------------------------------
%\usepackage{kpfonts}
%\usepackage{titletoc}
\contentsmargin{0cm}
\titlecontents{chapter}[0pc]
{\addvspace{30pt}%
\begin{tikzpicture}[remember picture, overlay]%
\draw[fill=white,draw=white] (-4,-.1) rectangle (0.0,0.5);% rectángulo antes del "capítulo tal"
\pgftext[left,x=-1.5 cm,y=0.2cm]{\color{verdeF}\Huge\sc\bfseries \ \thecontentslabel}%
\end{tikzpicture}\color{verdeF}\large\sc\bfseries}%
{}
{}
{\;\titlerule\;\large\sc\bfseries Página \thecontentspage
\begin{tikzpicture}[remember picture, overlay]
\draw[fill=white,draw=verdeF] (2pt,0) rectangle (6,0.1pt);
\end{tikzpicture}}%
\titlecontents{section}[2.4pc]
{\addvspace{1pt}}
{\contentslabel[\color{verdeF}\thecontentslabel]{2.4pc}}
{}
{\hfill\small \color{azulF}\thecontentspage}
[]
\titlecontents*{subsection}[4pc]
{\addvspace{-1pt}\small}
{}
{}
{\hfill\small \color{verdeF}\thecontentspage}
[ \textbullet\ ][]

\makeatletter
\renewcommand{\tableofcontents}{%
%Título: Indice General
\chapter*{\contentsname}%
\@starttoc{toc}}
\makeatother

% Fin Contenido -------------------------------------------------------------------

%----------------------------------------------------------------------------------------
%	Numeración de las secciones -- en el margen
%----------------------------------------------------------------------------------------

\makeatletter
\renewcommand{\@seccntformat}[1]{\llap{\textcolor{morado}{\csname the#1\endcsname}\hspace{1ex}}}                    
\renewcommand{\section}{\@startsection{section}
{1}{0.8cm}%0.9 corre la numeración hacia adentro
{-4ex \@plus -1ex \@minus -.4ex}
{1ex \@plus.2ex }
{\color{azulF}\normalfont\Large\sffamily\bfseries}}
\renewcommand{\subsection}{\@startsection {subsection}{2}{\z@}
{-3ex \@plus -0.1ex \@minus -.4ex}
{0.5ex \@plus.2ex }
{\color{azulF}\normalfont\large\sffamily\bfseries}}
\renewcommand{\subsubsection}{\@startsection {subsubsection}{3}{\z@}
{-2ex \@plus -0.1ex \@minus -.2ex}
{0.2ex \@plus.2ex }
{\color{azulF}\normalfont\sffamily\bfseries}}                        
\renewcommand\paragraph{\@startsection{paragraph}{4}{\z@}
{-2ex \@plus-.2ex \@minus .2ex}
{0.1ex}
{\normalfont\small\sffamily\bfseries}}
\makeatother
% Fin numeración secciones
