%% Configuracion generales para entornos
% Contadores
\newcounter{tcbteo}[chapter]
\renewcommand{\thetcbteo}{\thechapter.\arabic{tcbteo}}

\newcounter{tcbdefi}[chapter]
\renewcommand{\thetcbdefi}{\thechapter.\arabic{tcbdefi}}

\newcounter{tcblema}[chapter]
\renewcommand{\thetcblema}{\thechapter.\arabic{tcblema}}

\newcounter{tcbcoro}[chapter]
\renewcommand{\thetcbcoro}{\thechapter.\arabic{tcbcoro}}

\newcounter{tcbpropo}[chapter]
\renewcommand{\thetcbpropo}{\thechapter.\arabic{tcbpropo}}

\newcounter{tcbvoca}[chapter]
\renewcommand{\thetcbvoca}{\thechapter.\arabic{tcbvoca}}

\newcounter{tcbejem}[chapter]
\renewcommand{\thetcbejem}{\thechapter.\arabic{tcbejem}}

\newlength{\examlen}

\tikzset{
    wnodeTeorema/.style={%
         rectangle,  top color=gray!5, bottom color=gray!5,
         inner sep=1mm,anchor=west,font=\bf\sffamily},
   wnodeminimo/.style={%
         rectangle,  top color=white, bottom color=white,
         text=azulF,inner sep=1mm,anchor=west,font=\bf\sffamily}      
}

% Entorno para contenidos, objetivos y conocimientos previos.
\newtcolorbox{informacion}[2][]{enhanced,
before skip=2mm,after skip=2mm,
colback=black!5,colframe=black!50,boxrule=0.2mm,
attach boxed title to top left={xshift=1cm,yshift*=1mm-\tcboxedtitleheight},
varwidth boxed title*=-3cm,
boxed title style={frame code={
\path[fill=tcbcolback!10!black]
([yshift=-1mm,xshift=-1mm]frame.north west)
arc[start angle=0,end angle=180,radius=1mm]
([yshift=-1mm,xshift=1mm]frame.north east)
arc[start angle=180,end angle=0,radius=1mm];
\path[left color=tcbcolback!80!black,right color=tcbcolback!80!black,
middle color=tcbcolback!95!black]
([xshift=-2mm]frame.north west) -- ([xshift=2mm]frame.north east)
[rounded corners=1mm]-- ([xshift=1mm,yshift=-1mm]frame.north east)
-- (frame.south east) -- (frame.south west)
-- ([xshift=-1mm,yshift=-1mm]frame.north west)
[sharp corners]-- cycle;
},interior engine=empty,
},
fonttitle=\bfseries,
title={#2},#1}

% Entorno con anillos color verde (posiblemente cambiar color)
\newlist{cirlist}{itemize}{1}
\setlist[cirlist]{
    label = {\wverde \faCircle[regular]}
}


% Entorno para  Definiciones---------------------------------------------------
\newtcolorbox{wwdefinicion}[3][]{%
arc=0mm,breakable,enhanced,colback=gray!5,boxrule=0pt,drop fuzzy shadow,
top=6mm,fontupper=\normalsize,step and label={tcbdefi}{#3},
overlay unbroken  = {
%barra vertical
\draw[color=colordominanteF,line width=3pt] ([xshift=2pt] frame.north west)--([xshift=2pt] frame.south west);        
%Caja de Título: defi --
\node[wnodeTeorema](titulodefi) at ([xshift=4.5pt, yshift=-3mm]frame.north west)
{\textbf{Definición \thetcbdefi \;#2}};
                }, %overlay
overlay first  = {
%barra vertical
\draw[color=colordominanteF,line width=3pt] ([xshift=2pt] frame.north west)--([xshift=2pt] frame.south west);        
%Caja de Título: defi --
\node[wnodeTeorema](titulodefi) at ([xshift=4.5pt, yshift=-3mm]frame.north west)
{\textbf{Definición \thetcbdefi \;#2}};
                }, %overlay
% Mantener borde en cambio de página
overlay last    = {%barra vertical
\draw[color=colordominanteF,line width=3pt] ([xshift=2pt] frame.north west)--([xshift=2pt] frame.south west);}
#1}
%-
\NewDocumentEnvironment{definicion}{O{} O{} O{}}{\smallskip\begin{wwdefinicion}{#1}{#2}%
 #3}{\end{wwdefinicion}\smallskip }
% %DEFINICION---------------------------------------------------------

% Entorno personalizado---------------------------------------------------
\definecolor{colorejercicios}{RGB}{99,42,134}

\newtcolorbox{wwejemplo}[3][]{%
arc=0mm,breakable,%drop fuzzy shadow,
enhanced,colback=gray!5,boxrule=0pt,top=7mm,
fontupper=\normalsize,step and label={tcbejem}{#3},
overlay unbroken = {
%Borde grueso superior
\draw[color=colorejercicios,line width=3pt] (frame.north west)--([xshift=0pt]frame.north east);
%Caja de Titulo: Ejer --
\node[rounded corners=3pt,  draw=colorejercicios, top color=white, bottom color=white, thick,inner sep=1mm,anchor=west, font=\bf\sffamily](tituloejem) at ([xshift=5mm, yshift=0mm]frame.north west)
{\textbf{\color{miverde}  Ejemplo \thetcbejem \;#2}};
%borde linea inferior
 \draw[color=colorejercicios,line width=0.2pt] (frame.south west)--([xshift=0pt]frame.south east); 
},%overlay
overlay first = {
%Borde grueso superior
\draw[color=colorejercicios,line width=3pt] (frame.north west)--([xshift=0pt]frame.north east);
%Caja de T�tulo: Ejer --
\node[rounded corners=3pt,  draw=colorejercicios, top color=white, bottom color=white, thick,inner sep=1mm,anchor=west, font=\bf\sffamily](tituloejem) at ([xshift=5mm, yshift=0mm]frame.north west)
{\textbf{\color{miverde}  Ejemplo \thetcbejem \;#2}};
%borde l�nea inferior
 \draw[color=colorejercicios,line width=0.2pt] (frame.south west)--([xshift=0pt]frame.south east); 
},%overlay
% % Mantener borde en cambio de p�gina 
% overlay middle = {\draw[color=colordominante,line width=0.2pt] (frame.north west)--([xshift=0pt]frame.north east);
%                 } 
overlay middle ={},
overlay last = { %borde l�nea inferior
 \draw[color=colorejercicios,line width=0.2pt] (frame.south west)--([xshift=0pt]frame.south east); 
                } 
#1}
%-
\NewDocumentEnvironment{ejemplo}{O{} O{} O{}}{\smallskip\begin{wwejemplo}{#1}{#2}%
 #3}{\end{wwejemplo}\smallskip }
%


% Entorno para notas y/o aclaraciones 
\newtcolorbox{notabox}[1][]{%
arc=0mm,breakable,
enhanced,colback=white,
boxrule=0pt,
top=3mm, %Separación vertical - inicia texto
left=25pt,
enlarge top by=\baselineskip/2+1mm,
enlarge top at break by=0mm,pad at break=2mm,
fontupper={\begin{tikzpicture}[overlay]
\node[draw=colordominanteF,line width=1pt,circle,fill=white,font=\sffamily\bfseries,inner sep=2pt,outer sep=0pt] at (-15pt,3pt){\textcolor{colordominanteF}{N}};\end{tikzpicture}}~\normalfont,  %"NOTA..."+texto del cuerpo
%Borde y círculo
overlay first={
\draw[color=white,line width=0.5pt] (frame.north west)
  --([xshift=0pt]frame.north east)
  --([xshift=0pt]frame.south east)
  --([xshift=0pt]frame.south west)--(frame.north west);
        },
%Borde y círculo
overlay first={
\draw[color=white,line width=0.5pt] (frame.north west)
  --([xshift=0pt]frame.north east)
  --([xshift=0pt]frame.south east)
  --([xshift=0pt]frame.south west)--(frame.north west);
        },
%Borde cambio de página
overlay last={\draw[color=white,line width=0.5pt] (frame.north west)
  --([xshift=0pt]frame.north east)
  --([xshift=0pt]frame.south east)
  --([xshift=0pt]frame.south west)--(frame.north west);}
#1}
%-
 \newenvironment{nota}[1][]{\bigskip\begin{notabox}%
 #1}{\end{notabox}}	
% Fin nota

% Teorema -----------------------------------------------------
\newtcolorbox{wwteorema}[3][]{%
arc=0mm,breakable,enhanced,colback=gray!5,boxrule=0pt,top=7mm,drop fuzzy shadow,
fontupper=\normalsize,step and label={tcbteo}{#3},
overlay unbroken = {\draw[color=colordominanteF,line width=0.2pt] (frame.north west)--([xshift=0pt]frame.north east);
%Caja de Título: teo --
\node[wnodeTeorema](tituloteo) at ([xshift=0pt, yshift=-4mm]frame.north west)
{\textbf{\color{colordominanteF} Teorema \thetcbteo \;#2}};
%Borde superior --
\draw[colordominanteF,line width=2.5cm] ([xshift=1.25cm, yshift=0cm]frame.north west)-- +(\examlen,3pt);
},%
overlay first = {\draw[color=colordominanteF,line width=0.2pt] (frame.north west)--([xshift=0pt]frame.north east);
%Caja de Título: teo --
\node[wnodeTeorema](tituloteo) at ([xshift=0pt, yshift=-4mm]frame.north west)
{\textbf{\color{colordominanteF} Teorema \thetcbteo \;#2}};
%Borde superior --
\draw[colordominanteF,line width=2.5cm] ([xshift=1.25cm, yshift=0cm]frame.north west)-- +(\examlen,3pt);
},%
% Mantener borde en cambio de página 
overlay last = {\draw[color=colordominanteF,line width=0.2pt] (frame.south west)--([xshift=0pt]frame.south east);
                } 
#1}
%-
\NewDocumentEnvironment{teorema}{O{} O{} O{}}{\smallskip\begin{wwteorema}{#1}{#2}%
 #3}{\end{wwteorema}\smallskip }
% TEOREMA---------------------------------------------------------